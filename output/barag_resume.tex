\documentclass[10pt]{barag_resume}

\author{Sean Barag}
\homeaddress{Seattle, WA}
\phonenumber{+1(215)-380-8844}
\emailaddress{sean@barag.org}
\portfolio{http://github.com/sjbarag}

\begin{document}
\section{Experience}\relax
    \renewcommand{\labelitemi}{$\cdot$}\relax % use cdot as a bullet.  it's less obvious
    \begin{job}{Hulu, LLC}
        {Seattle, WA}
        {Interim Software Development Manager --- Living Room}
        {job_0}
    \end{job}

    \begin{job}{Hulu, LLC}
        {Seattle, WA}
        {Principal Software Developer --- Living Room}
        {job_1}
            \begin{itemize*}
                \item Led team of 7 devs through complete UI rewrite, from prototyping to launch
                \item Implemented Redux pattern in Roku ecosystem
                \item Continued public maintenance of Hulu open-source projects (brs, roca, linter)
                \item Integrated OSS browser automation framework for automated UI testing
                \item Formal mentor to two other devs at Hulu
            \end{itemize*}
    \end{job}

    \begin{job}{Hulu, LLC}
        {Seattle, WA}
        {Senior Software Developer --- Living Room}
        {job_2}
            \begin{itemize*}
                \item Maintained Hulu app on Roku, written in Roku-proprietary language
                \item Led team of 5 devs to launch Hulu on Amazon Echo Show
                \item Released splitshot (see open-source contributions) in public, including Hulu Tech Blog post
                \item Integrated device-free unit testing for Roku app
                \item Became internal point of contact for all Hulu open-source releases
            \end{itemize*}
    \end{job}

    \begin{job}{Hulu, LLC}
        {Seattle, WA}
        {Software Developer --- Living Room}
        {job_3}
            \begin{itemize*}
                \item Maintained HTML/CSS/JS Hulu app on consoles, smart TVs, etc.
                \item Worked with team of 4 to launch profiles support in HTML app
                \item Embedded with backend team to build initial version of client-facing content API for live TV
                \item Investigated app performance for Hulu with Live TV launch
                \item Implemented view-model pattern in backend and in Hulu on Roku
            \end{itemize*}
    \end{job}

    \begin{job}{Amazon.com}
        {Seattle, WA}
        {Software Development Engineer II --- Marketplace Shipping Services}
        {job_4}
            \begin{itemize*}
                \item Designed, implemented, and launched new public web service to purchase shipping labels for orders
                \item Supported external users during six months of private beta of shipping label web service
                \item Coordinated team of six developers working simultaneously in new codebase
                \item Supported and advised team of India-based developers for contributions to my team's codebase
            \end{itemize*}
    \end{job}

    \begin{job}{Amazon.com}
        {Seattle, WA}
        {Software Development Engineer I --- Marketplace Shipping Services}
        {job_5}
            \begin{itemize*}
                \item Reduced fatal errors from 1,100 per million requests to 200
                \item Contributed to internal tools to read and transform log data
                \item Developed and maintained Java service capable of processing \~{}1,000 requests per-second (peak)
                \item Designed asynchronous workflow-based service to purchase up to 500 shipping labels merged into single PDF
            \end{itemize*}
    \end{job}

    \begin{job}{Lockheed Martin MS2}
        {Moorestown, NJ}
        {Senior Technical Co-Op --- Information Assurance (IA)}
        {job_6}
    \end{job}

    \begin{job}{Lockheed Martin MS2}
        {Moorestown, NJ}
        {Senior Technical Co-Op --- Computer Infrastructure \& Integration Testing}
        {job_7}
    \end{job}

    \begin{tikzpicture}[overlay, transform canvas={xshift=-0.5ex}]
        \node [bubble] at (job_0) (job_0_bubble) {};
        \node [align=right, text width=3cm, base left=2ex of job_0_bubble.base]
            at (job_0)
            (job_0_label)
            {\small Jun 2020\\ --- Present};
        \node [bubble] at (job_1) (job_1_bubble) {};
        \node [align=right, text width=3cm, base left=2ex of job_1_bubble.base]
            at (job_1)
            (job_1_label)
            {\small Mar 2020\\ --- Jun 2021};
        \node [bubble] at (job_2) (job_2_bubble) {};
        \node [align=right, text width=3cm, base left=2ex of job_2_bubble.base]
            at (job_2)
            (job_2_label)
            {\small Mar 2018\\ --- Mar 2020};
        \node [bubble] at (job_3) (job_3_bubble) {};
        \node [align=right, text width=3cm, base left=2ex of job_3_bubble.base]
            at (job_3)
            (job_3_label)
            {\small Mar 2016\\ --- Mar 2018};
        \node [bubble] at (job_4) (job_4_bubble) {};
        \node [align=right, text width=3cm, base left=2ex of job_4_bubble.base]
            at (job_4)
            (job_4_label)
            {\small Apr 2015\\ --- Mar 2016};
        \node [bubble] at (job_5) (job_5_bubble) {};
        \node [align=right, text width=3cm, base left=2ex of job_5_bubble.base]
            at (job_5)
            (job_5_label)
            {\small Aug 2013\\ --- Apr 2015};
        \node [bubble] at (job_6) (job_6_bubble) {};
        \node [align=right, text width=3cm, base left=2ex of job_6_bubble.base]
            at (job_6)
            (job_6_label)
            {\small Apr 2012\\ --- Sep 2012};
        \node [bubble] at (job_7) (job_7_bubble) {};
        \node [align=right, text width=3cm, base left=2ex of job_7_bubble.base]
            at (job_7)
            (job_7_label)
            {\small Mar 2011\\ --- Sep 2011};
        \draw
            (job_0_bubble) --
            (job_1_bubble) --
            (job_2_bubble) --
            (job_3_bubble) --
            (job_4_bubble) --
            (job_5_bubble) --
            (job_6_bubble) --
            (job_7_bubble);
    \end{tikzpicture}

    \newpage % because of the tikz-rendered bubbles, side projects **must** be on one page!
\section{Notable\\ Side Projects}\relax
    \begin{project}
        {thcon}
        {A configurable CLI tool that switches multiple apps between light and dark themes}
        {proj_0}
            Goal: See my screen better in the sun with fewer steps, teach myself Rust\\
            Status: Published and actively developed \\ 
            URL: https://github.com/theme-controller/thcon
    \end{project}

    \begin{project}
        {brs}
        {BrightScript interpreter implemented in TypeScript}
        {proj_1}
            Goal: Enable off-device testing by implementing (from-scratch) the language Roku apps are written in\\
            Status: Maintained with coworkers and in active use \\ 
            URL: https://github.com/sjbarag/brs
    \end{project}

    \begin{project}
        {PNRG (Poorly Named Resume Generator)}
        {Templated resume source generator}
        {proj_2}
            Goal: Reduce effort required to maintain multiple parallel resume formats\\
            Status: Frozen; updated as-needed \\ 
            URL: https://github.com/sjbarag/poorly-named-resume-generator
    \end{project}

    \begin{project}
        {Textspansion}
        {Text expansion for Android devices}
        {proj_3}
            Goal: Add missing features to Android ecosystem while learning Android dev paradigms\\
            Status: Released June 2011, featured on Lifehacker.com in Sept 2011, now deprecated    \end{project}

    \begin{tikzpicture}[overlay, transform canvas={xshift=-0.5ex}]
        \node [bubble] at (proj_0) (proj_0_bubble) {};
        \node [align=right, text width=3cm, base left=2ex of proj_0_bubble.base]
            at (proj_0)
            (proj_0_label)
            {\small Oct 2020};
        \node [bubble] at (proj_1) (proj_1_bubble) {};
        \node [align=right, text width=3cm, base left=2ex of proj_1_bubble.base]
            at (proj_1)
            (proj_1_label)
            {\small Dec 2017};
        \node [bubble] at (proj_2) (proj_2_bubble) {};
        \node [align=right, text width=3cm, base left=2ex of proj_2_bubble.base]
            at (proj_2)
            (proj_2_label)
            {\small Aug 2015};
        \node [bubble] at (proj_3) (proj_3_bubble) {};
        \node [align=right, text width=3cm, base left=2ex of proj_3_bubble.base]
            at (proj_3)
            (proj_3_label)
            {\small May 2011};
        \draw
            (proj_0_bubble) --
            (proj_1_bubble) --
            (proj_2_bubble) --
            (proj_3_bubble);
    \end{tikzpicture}

\section{Notable\\ Open-Source\\ Contributions}\relax
    \begin{project}
        {git}
        {Added support for clone.defaultRemoteName config}
        {cont_0}
            Status: Submitted and released in 2.30\\
            URL: \url{https://github.com/git/git/compare/54e85e7af1...de9ed3ef37}
    \end{project}

    \begin{project}
        {roca}
        {Roku unit testing framework implemented on top of brs}
        {cont_1}
            Status: Developed in public, released, actively maintained with peers\\
            URL: \url{https://github.com/hulu/roca}
    \end{project}

    \begin{project}
        {splitshot}
        {Generates TypeScript declarations from CoffeeScript sources}
        {cont_2}
            Status: Developed internally, released, frozen\\
            URL: \url{https://github.com/hulu/splitshot}
    \end{project}

    \begin{project}
        {TypeScript}
        {Automatically inherit JSDoc comments subclasses and subinterfaces}
        {cont_3}
            Status: Submitted and released in 2.7\\
            URL: \url{https://github.com/microsoft/TypeScript/pull/18804}
    \end{project}

    \begin{tikzpicture}[overlay, transform canvas={xshift=-0.5ex}]
        \node [bubble] at (cont_0) (cont_0_bubble) {};
        \node [align=right, text width=3cm, base left=2ex of cont_0_bubble.base]
            at (cont_0)
            (cont_0_label)
            {\small Sep 2020};
        \node [bubble] at (cont_1) (cont_1_bubble) {};
        \node [align=right, text width=3cm, base left=2ex of cont_1_bubble.base]
            at (cont_1)
            (cont_1_label)
            {\small Jan 2019};
        \node [bubble] at (cont_2) (cont_2_bubble) {};
        \node [align=right, text width=3cm, base left=2ex of cont_2_bubble.base]
            at (cont_2)
            (cont_2_label)
            {\small Jan 2018};
        \node [bubble] at (cont_3) (cont_3_bubble) {};
        \node [align=right, text width=3cm, base left=2ex of cont_3_bubble.base]
            at (cont_3)
            (cont_3_label)
            {\small Sep 2017};
        \draw
            (cont_0_bubble) --
            (cont_1_bubble) --
            (cont_2_bubble) --
            (cont_3_bubble);
    \end{tikzpicture}

\section{Education}\relax
    {\large\textbf{Drexel University}} --- Graduated Jun 2013 \hfill{\small Philadelphia, PA}\\
    BS in Computer Engineering \hfill Cumulative GPA: 3.58 $|$ In-major: 3.63\\
    BS in Electrical Engineering (Electronics)\\
    Minor in Computer Science\\

\section{Skills}\relax
    \vspace{-2em} % ensure the top line of the multicol environment aligns with the section title
    \begin{multicols}{2}
        \begin{csitemize}{Languages}
            \item Bash/POSIX shells
            \item C
            \item Java
            \item \textrm{\LaTeX}
            \item Python
            \item Perl (forever rusty)
            \item Rust (casual observer)
        \end{csitemize}

        \begin{csitemize}{Operating Systems}
            \item Linux (Arch, Debian, Fedora, RHEL, Ubuntu)
            \item Unix (Solaris 9)
            \item Mac OS (10.5, 10.6)
        \end{csitemize}

        \begin{csitemize}{Tools}
            \item vim
            \item GNU coreutils
            \item git, mercurial
            \item Spring
            \item Django
            \item AWS S3, DyamoDB, SQS, SNS
        \end{csitemize}

        \begin{csitemize}{Paradigms}
            \item Distributed systems
            \item Service-Oriented Architectures (SOA)
            \item SQL and NoSQL databases
        \end{csitemize}
    \end{multicols}
\end{document}