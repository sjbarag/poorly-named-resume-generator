\documentclass[10pt]{barag_resume}

\title{Hire Sean Barag}
\author{Sean Barag}
\homeaddress{Seattle-ish, WA}
\phonenumber{+1215-380-8844}
\emailaddress{sean@barag.org}
\portfolio{https://github.com/sjbarag}

\begin{document}
\marginsection{Experience}\relax
    \renewcommand{\labelitemi}{$\cdot$}\relax % use cdot as a bullet.  it's less obvious
    \begin{job}{Cockroach Labs}
        {NYC (Remote)}
        {Senior Software Developer --- Cloud \& Developer Experiences}
        {job_0}
            \begin{itemize*}
                \item Responsible for UI development tools, CI/CD infra, build systems (Bazel and node-native)
                \item Built automatic preview deployments for all UI PRs (\~{}self-hosted Vercel)
                \item Benchmarked JSON vs multiple protobuf clients with real-world payloads in all major browsers
                \item Reduced CockroachDB UI build times by > 3 minutes
                \item Set up real-user monitoring and structured error logging in SaaS UI
                \item Maintained React + Redux SaaS web UI
            \end{itemize*}
    \end{job}

    \begin{job}{Hulu}
        {Seattle, WA}
        {Principal Software Developer --- Living Room}
        {job_1}
            \begin{itemize*}
                \item Led team of 7 devs through complete UI rewrite, from prototyping to launch
                \item Implemented Redux pattern in Roku ecosystem
                \item Continued public maintenance of Hulu open-source projects (brs, roca, linter)
                \item Integrated OSS browser automation framework for webapp UI testing
                \item Formal mentor to two other devs
            \end{itemize*}
    \end{job}

    \begin{job}{Hulu}
        {Seattle, WA}
        {Senior Software Developer --- Living Room}
        {job_2}
            \begin{itemize*}
                \item Maintained Hulu app on Roku, written in Roku-proprietary language
                \item Led team of 5 devs to launch Hulu on Amazon Echo Show
                \item Released splitshot (see open-source contributions) to the public, including Hulu Tech Blog post
                \item Integrated device-free unit testing for Roku app
                \item Became internal point of contact for all Hulu open-source releases
            \end{itemize*}
    \end{job}

    \begin{job}{Hulu}
        {Seattle, WA}
        {Software Developer --- Living Room}
        {job_3}
            \begin{itemize*}
                \item Maintained HTML/CSS/JS Hulu app on consoles, smart TVs, etc.
                \item Worked with team of 4 to launch profiles support in HTML app
                \item Embedded with backend team to build initial version of client-facing content API for live TV
                \item Investigated app performance for Hulu with Live TV launch
                \item Implemented view-model pattern in backend and in Hulu on Roku
            \end{itemize*}
    \end{job}

    \begin{job}{Amazon.com}
        {Seattle, WA}
        {Software Development Engineer II --- Marketplace Shipping Services}
        {job_4}
            \begin{itemize*}
                \item Designed, implemented, and launched public API to purchase shipping labels for orders
                \item Supported external users during six months of private beta of shipping label web service
                \item Coordinated team of six developers working simultaneously in new codebase
                \item Supported and advised team of India-based developers for contributions to my team's codebase
            \end{itemize*}
    \end{job}

    \begin{job}{Amazon.com}
        {Seattle, WA}
        {Software Development Engineer I --- Marketplace Shipping Services}
        {job_5}
            \begin{itemize*}
                \item Reduced fatal errors from 1,100 per million requests to 200
                \item Contributed to internal tools to read and transform log data
                \item Developed and maintained Java service capable of processing \~{}1,000 requests per-second (peak)
                \item Designed async workflow-based service to buy up to 500 shipping labels \& merge into one PDF
            \end{itemize*}
    \end{job}

    \begin{job}{Drexel University Capstone Project}
        {Philadelphia, PA}
        {Embedded Software Developer --- Waterproof Recording Ammeter}
        {job_6}
            \begin{itemize*}
                \item Designed supporting circuits for PIC 18F microontroller and sensors
                \item Implemented I2C sensor readings via vendor library
                \item Implemented data persistence on SD card via SPI via vendor library
                \item Built USB serial support for configuration via PC
            \end{itemize*}
    \end{job}

    \begin{tikzpicture}[overlay, transform canvas={xshift=-0.5ex}]
        \node [bubble] at (job_0) (job_0_bubble) {};
        \node [align=right, text width=3cm, base left=2ex of job_0_bubble.base]
            at (job_0)
            (job_0_label)
            {\small Oct 2021\\ --- Present};
        \node [bubble] at (job_1) (job_1_bubble) {};
        \node [align=right, text width=3cm, base left=2ex of job_1_bubble.base]
            at (job_1)
            (job_1_label)
            {\small Mar 2020\\ --- Oct 2021};
        \node [bubble] at (job_2) (job_2_bubble) {};
        \node [align=right, text width=3cm, base left=2ex of job_2_bubble.base]
            at (job_2)
            (job_2_label)
            {\small Mar 2018\\ --- Mar 2020};
        \node [bubble] at (job_3) (job_3_bubble) {};
        \node [align=right, text width=3cm, base left=2ex of job_3_bubble.base]
            at (job_3)
            (job_3_label)
            {\small Mar 2016\\ --- Mar 2018};
        \node [bubble] at (job_4) (job_4_bubble) {};
        \node [align=right, text width=3cm, base left=2ex of job_4_bubble.base]
            at (job_4)
            (job_4_label)
            {\small Apr 2015\\ --- Mar 2016};
        \node [bubble] at (job_5) (job_5_bubble) {};
        \node [align=right, text width=3cm, base left=2ex of job_5_bubble.base]
            at (job_5)
            (job_5_label)
            {\small Aug 2013\\ --- Apr 2015};
        \node [bubble] at (job_6) (job_6_bubble) {};
        \node [align=right, text width=3cm, base left=2ex of job_6_bubble.base]
            at (job_6)
            (job_6_label)
            {\small Sep 2012\\ --- Jun 2013};
        \draw
            (job_0_bubble) --
            (job_1_bubble) --
            (job_2_bubble) --
            (job_3_bubble) --
            (job_4_bubble) --
            (job_5_bubble) --
            (job_6_bubble);
    \end{tikzpicture}

    \newpage % because of the tikz-rendered bubbles, side projects **must** be on one page!
\marginsection{Notable\\ Side Projects}\relax
    \begin{project}
        {caiac}
        {Code as Infrastructure as Code}
        {proj_0}
            Goal: Have a laugh by writing a Terraform provider that models the Go AST and emits .go files\\
            Status: Published POC and frozen \\ 
            \url{https://github.com/sjbarag/caiac}
    \end{project}

    \begin{project}
        {thcon}
        {A command-line tool + config file that switches themes in many apps}
        {proj_1}
            Goal: Ease eye-strain by toggling dark mode in multiple tools with fewer steps\\
            Status: Published and actively maintained \\ 
            \url{https://thcon.app}
    \end{project}

    \begin{project}
        {brs}
        {BrightScript interpreter implemented in TypeScript}
        {proj_2}
            Goal: Enable off-device testing by implementing (from-scratch) the language Roku apps are written in\\
            Status: In use but frozen; forked by Roku dev community \\ 
            \url{https://github.com/sjbarag/brs}
    \end{project}

    \begin{project}
        {PNRG (Poorly Named Resume Generator)}
        {Templated resume source generator}
        {proj_3}
            Goal: Reduce effort required to maintain multiple parallel resume formats\\
            Status: Frozen; updated as-needed \\ 
            \url{https://github.com/sjbarag/poorly-named-resume-generator}
    \end{project}

    \begin{tikzpicture}[overlay, transform canvas={xshift=-0.5ex}]
        \node [bubble] at (proj_0) (proj_0_bubble) {};
        \node [align=right, text width=3cm, base left=2ex of proj_0_bubble.base]
            at (proj_0)
            (proj_0_label)
            {\small Feb 2023};
        \node [bubble] at (proj_1) (proj_1_bubble) {};
        \node [align=right, text width=3cm, base left=2ex of proj_1_bubble.base]
            at (proj_1)
            (proj_1_label)
            {\small Oct 2020};
        \node [bubble] at (proj_2) (proj_2_bubble) {};
        \node [align=right, text width=3cm, base left=2ex of proj_2_bubble.base]
            at (proj_2)
            (proj_2_label)
            {\small Dec 2017};
        \node [bubble] at (proj_3) (proj_3_bubble) {};
        \node [align=right, text width=3cm, base left=2ex of proj_3_bubble.base]
            at (proj_3)
            (proj_3_label)
            {\small Aug 2015};
        \draw
            (proj_0_bubble) --
            (proj_1_bubble) --
            (proj_2_bubble) --
            (proj_3_bubble);
    \end{tikzpicture}

\marginsection{Notable\\ Open-Source\\ Contributions}\relax
    \begin{project}
        {git}
        {Added support for clone.defaultRemoteName config}
        {cont_0}
            Status: Submitted and released in 2.30\\
            \url{https://github.com/git/git/compare/54e85e7af1...de9ed3ef37}
    \end{project}

    \begin{project}
        {roca}
        {Roku unit testing framework implemented on top of brs}
        {cont_1}
            Status: Developed in public and released; abandoned by Hulu after I left\\
            \url{https://github.com/hulu/roca}
    \end{project}

    \begin{project}
        {splitshot}
        {Generates TypeScript declarations from CoffeeScript sources}
        {cont_2}
            Status: Developed internally, released, frozen\\
            \url{https://github.com/hulu/splitshot}
    \end{project}

    \begin{project}
        {TypeScript}
        {Automatically inherit JSDoc comments in subclasses and subinterfaces}
        {cont_3}
            Status: Submitted and released in 2.7\\
            \url{https://github.com/microsoft/TypeScript/pull/18804}
    \end{project}

    \begin{tikzpicture}[overlay, transform canvas={xshift=-0.5ex}]
        \node [bubble] at (cont_0) (cont_0_bubble) {};
        \node [align=right, text width=3cm, base left=2ex of cont_0_bubble.base]
            at (cont_0)
            (cont_0_label)
            {\small Sep 2020};
        \node [bubble] at (cont_1) (cont_1_bubble) {};
        \node [align=right, text width=3cm, base left=2ex of cont_1_bubble.base]
            at (cont_1)
            (cont_1_label)
            {\small Jan 2019};
        \node [bubble] at (cont_2) (cont_2_bubble) {};
        \node [align=right, text width=3cm, base left=2ex of cont_2_bubble.base]
            at (cont_2)
            (cont_2_label)
            {\small Jan 2018};
        \node [bubble] at (cont_3) (cont_3_bubble) {};
        \node [align=right, text width=3cm, base left=2ex of cont_3_bubble.base]
            at (cont_3)
            (cont_3_label)
            {\small Sep 2017};
        \draw
            (cont_0_bubble) --
            (cont_1_bubble) --
            (cont_2_bubble) --
            (cont_3_bubble);
    \end{tikzpicture}

\marginsection{Skills}\relax
    \vspace{-2em} % ensure the top line of the multicol environment aligns with the section title
    \begin{multicols}{2}
        \begin{csitemize}{Languages}
            \item Bash/POSIX shells
            \item Docker (as needed)
            \item Go
            \item JavaScript / TypeScript
            \item \textrm{\LaTeX}
            \item Python
            \item Rust (still learning)
        \end{csitemize}

        \begin{csitemize}{Particularly Excited About}
            \item Building simple, powerful tooling
            \item Debugging (weird bugs are fun!)
            \item Fast, reliable, understandable tests
            \item Helping developers grow
            \item Performance optimization
        \end{csitemize}

        \begin{csitemize}{Tools}
            \item AWS S3, DyamoDB, SQS, SNS, Lambda
            \item Bazel
            \item git
            \item HTTP debugging proxies (Charles, Proxyman)
            \item Kubernetes (as needed)
            \item React/Redux
            \item Terraform
        \end{csitemize}

        \begin{csitemize}{Paradigms}
            \item Custom linters
            \item Distributed systems
            \item HTML Single-Page Apps (SPA)
            \item HTTP REST (client and server)
            \item Language interpreters
        \end{csitemize}
    \end{multicols}
\end{document}